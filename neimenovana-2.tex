\documentclass[A4]{article}
\usepackage{amsmath}
\usepackage{listingsutf8}
\lstset{literate={č}{{\v c}}1 {š}{{\v s}}1 {ž}{{\v z}}1} %Paket listings (oz. listingsutf8) direktno ne podpirata šumnikov.
\lstset{basicstyle=\ttfamily, language=Octave} %Nastavimo, da je privzet jezik za lstlistings octave.
\usepackage[pdftex]{graphicx}
\usepackage[slovene]{babel} % slovenske nastavitve (naslovi, deljenje besed ...)
\usepackage[T1]{fontenc}    % font encoding; T1 podpira slovenščino
\usepackage[utf8]{inputenc} % input encoding; lahko je tudi [cp1250] ali [latin2]
\usepackage{verbatim}

\begin{document}
\title {Model drsečega povprečja z \emph{n} zakasnitvami}
\author{Nik Držanič \\  \\ 1. domača naloga pri Matematičnem modeliranju}
\maketitle
\section{Naloga}
S pomočjo matematičnega modela bi radi na podlagi N+1 merjenj napovedali obnašanje sistema, ki nam pri nekem vhodnem signalu x(t) vrne izhodni signal y(t).
\\ Signala x(t) in y(t) merimo ob časih t = 0, 1, ..., N. V modeliranju problema predpostavimo tudi, da lahko izhod y(t) ob času t zapišemo kot linearno kombinacijo n vhodov x(t), x(t - 1), ... , x(t - n + 1) iz časovnega intervala
 \\ $[t - n + 1, t]$ , t.j. 
\[
	y(t) = h_{1} x(t - n + 1) + h_{2} x(t - n + 2) + ... + h_{n-1}x(t-1) + h_{n}x(t). 
\]
\begin{flushleft}
Naše podatke tvori N+1 vhodov 
\end{flushleft}
{\centering
\[ 
x_{0},x_{1},...,x_{N}
\] \par
}
in N+1 pripadajočih izhodov 
{\centering
\[ 
y_{0},y_{1},...,y_{N}
\] \par
}
ki jih izmerimo pri celoštevilskih časih t iz intervala [0,N]. 
\section {Rešitev} 
\subsection {Postopek reševanja} 
Zgornjo enačbo preoblikujemo v 
{\centering
\[ 
y_{k} = h_{1} x_{k - n + 1} + h_{2} x_{k - n + 2} + ... + h_{n-1}x_{k - 1} + h_{n}x_{k}, 
\] \par
}
za vse k = n - 1, n, ..., N. Imamo torej sistem N - n + 1 enačb z n neznankami $h_{1} , ... , h_{n} :$ 
{\centering
\[ 
y_{n - 1} = h_{1} x_{0} + h_{2} x_{1} + ... + h_{n - 1}x_{n - 2} + h_{n}x_{n - 1}, 
\]
\[
y_{n} = h_{1} x_{1} + h_{2} x_{2} + ... + h_{n - 1}x_{n - 1} + h_{n}x_{n}, 
\]
\\ ...
\[ y_{N} = h_{1} x_{N - n + 1} + h_{2} x_{N - n + 2} + ... + h_{n - 1}x_{N - 1} + h_{n}x_{N}. 
\] \par }
Rešujemo sistem $Ah  = y$, kjer je 
\[
A = 
\begin{bmatrix}
x_{0} & x_{1} & \cdots  & x_{n - 2} & x_{n - 1}\\
x_{1} & x_{2} & \cdots & x_{n - 1} & x_{n}\\
\vdots  & \vdots  & \ddots & \vdots &  \vdots \\
x_{N - n + 1} & x_{N - n + 2} & \cdots & x_{N - 1} & x_{N}\\
\end{bmatrix}
\]
matrika našega sistema, $h = [h_{1}, ... , h_{n}]^T$ je naš vektor neznank, 
\[
y = 
\begin{bmatrix}
h_{1}x_{0} + h_{2}x_{1}+ \cdots  + h_{n - 1}x_{n - 2} + h_{n}x_{n - 1}\\
h_{1}x_{1} + h_{2}x_{2} + \cdots + h_{n - 1}x_{n - 1} + h_{n}x_{n}\\
\cdots \\
h_{1}x_{N - n + 1} + h_{2}x_{N - n + 2} + \cdots  + h_{n - 1}x_{N - 1} + h_{n}x_{N}\\
\end{bmatrix}
=
\begin{bmatrix}
y_{n-1}\\
y_{n}\\
\cdots \\
y_{N}\\
\end{bmatrix}
\]
pa je matrika izhodov.
\\[0.1in] Rešujemo torej sistem Ah = y v smislu linearne metode najmanjših kvadratov. Ob znanem h in znanimi vhodnimi podatki $x_{0}, ..., x_{m}$ bi tako lahko napovedali N - n + 1 členov izhoda sistema $y_{0}, ... , y_{m}$ , ki imajo indeks v intervalu [N - n + 1, N]. To storimo tako, da najprej izračunamo matriko A našega sistema ter to matriko pomnožimo z vektorjem neznank h.
\\[0.3in] Če so vsi vhodi in izhodi konstantni, 
\[
	x = [x_{0},x_{1},...,x_{N}], 
\]
\[
	y = [y_{0},y_{1},...,y_{N}],
\]
\[
	x_{i} = y_{j}, { \forall i,j = 0, 1, ..., N }, 
\]
\[
	x_{i} = x_{j}, { \forall i,j = 0, 1, ..., N }, i \neq j
\]
in nas zanima rešitev h sistema Ah = y  za poljubna N in n lahko za napoved komponent h vektorja uporabimo Moore-Penrose inverz matrike A(ker imamo več enačb kot pa neznank) 
\[
Ah = y
\] 
obe strani enačbe levo pomnožimo z Moore-Penrose inverzom matrike A
\[
A^{+}A h = A^{+}y
\]
vemo da je $A^{+}A = I$
\[ 
Ih = A^{+}y
\]
\[
h = A^{+}y
\]
{\centering
\[ 
y_{n - 1} = h_{1} x_{0} + h_{2} x_{1} + ... + h_{n - 1}x_{n - 2} + h_{n}x_{n - 1}, 
\]
\[
y_{n} = h_{1} x_{1} + h_{2} x_{2} + ... + h_{n - 1}x_{n - 1} + h_{n}x_{n}, 
\]
\\ ...
\[ y_{N} = h_{1} x_{N - n + 1} + h_{2} x_{N - n + 2} + ... + h_{n - 1}x_{N - 1} + h_{n}x_{N}. 
\] \par }
lahko prepišemo v 
{\centering
\[ 
y_{n - 1} = h_{1}y_{n-1} + h_{2}y_{n-1} + ... + h_{n - 1}y_{n-1} + h_{n}y_{n - 1}, 
\]
\[
y_{n} = h_{1}y_{n} + h_{2}y_{n} + ... + h_{n - 1}y_{n} + h_{n}y_{n}, 
\]
\\ ...
\[ y_{N} = h_{1} y_{N} + h_{2} y_{N} + ... + h_{n - 1}y_{N} + h_{n}y_{N}. 
\] 
in slednje poenostavimo v 
\[ 
y_{n - 1} = y_{n - 1} (h_{1} + h_{2} + ... + h_{n})  
\]
\[
y_{n} = y_{n} (h_{1} + h_{2} + ... + h_{n})  
\]
...
\[
y_{N} = y_{N} (h_{1} + h_{2} + ... + h_{n}) .
\]
vsako enačbo nato delimo s katerimkoli členom vhoda ali izhoda ter dobimo 
\[ 
1 = h_{1} + h_{2} + ... + h_{n}   
\]
\[
1 = h_{1} + h_{2} + ... + h_{n}  
\]
...
\[
1 = h_{1} + h_{2} + ... + h_{n} 
\] 
sistem N - n + 1 enačb lahko seštejemo v eno enačbo 
\[ 
N - n + 1 = (N - n + 1)h_{1} + (N - n + 1)h_{2} + ... + (N - n + 1)h_{n}
\]
kar poenostavimo v 
\[
1 = h_{1} + h_{2} + ... + h_{n} 
\]
ker so vhodi in izhodi konstantni, morajo biti  
\par }

\subsection {Implementacija v Octave}
V octave-u bomo implementirali funkciji movavg(x, y, n), ki za vhodne podatke $x = [x_{0}, ... , x_{N}]$ in izhodne podatke $y = [y_{0}, ... , y_{N}]$ poišče koeficiente $h = [h_{1}, ..., h_{n}]$ za model drsečega povprečja z n zakasnitvami ter funkcijo prediction(x, h), ki iz vhodnih podatkov x in koeficientov h napove izhod y na podlagi našega modela.
\subsection {Reference} 
https://www.youtube.com/watch?v=5bxsxM2UTb4 - uporaba Moore-Penrose inverza za reševanja sistema linearnih enačb

\end{document}